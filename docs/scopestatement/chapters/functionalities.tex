\section{Sensor Data Acquisition}
The system continuously collects vital health data using the integrated sensors. 
These include:
\begin{itemize}
    \item Heart rate and blood oxygen levels via the MAX30100 sensor,
    \item Body temperature using the DS18B20 sensor,
    \item Blood pressure measurements with the MPS20N0040D sensor,
    \item Environmental and situational awareness through the AI camera (or ESP32-CAM),
    \item Optional motion detection using a PIR sensor to activate the camera only when needed.
\end{itemize}

The raw data is processed locally on the Raspberry Pi for efficiency, and transmitted securely to the backend for storage and analysis.

\section{Mobile Application Development}
A mobile application will be developed as a Progressive Web App (PWA) combined with a Single Page Application (SPA). 
This will serve as the main user dashboard for both healthcare professionals and families monitoring elders at home.

The app will provide the following functionalities:
\begin{itemize}
    \item Real-time visualization of health vitals (heartbeat, SpO2, body temperature, blood pressure),
    \item Notifications and alerts in case of abnormal readings or critical events,
    \item GPS-based alert system in case of a fall detection or potential heart attack event,
    \item User-friendly interface accessible from smartphones, tablets, or desktop browsers without requiring installation.
\end{itemize}

\section{Machine Learning Integration}

\subsubsection{Model Development}
The goal is to develop a compact and efficient machine learning model tailored to analyze vital signs 
and/or video data from the ESP32-CAM (or AI CAM) and wearable sensors installed in a bracelet. 
Depending on the chosen approach, the model will perform:

\begin{itemize}
    \item \textbf{Fall Detection:} Analyzing video or motion sensor data to automatically detect falls 
    and trigger immediate alerts.
    \item \textbf{Heart Attack Prediction:} Using real-time vital sign data (heart rate, SpO2, blood pressure) 
    to predict potential cardiac events and notify caregivers promptly.
\end{itemize}

The model will leverage lightweight architectures suitable for embedded devices 
or edge computing, ensuring real-time performance and reliable detection 
even with the limited resources of the ESP32-CAM or local processing unit.

\subsubsection{MLOps Implementation}
An MLOps pipeline will streamline the training, deployment, and monitoring of the predictive models. 
Key features include:

\begin{itemize}
    \item \textbf{Automated Training and Deployment:} The MLOps framework will periodically train and update 
    the model with new sensor data, deploying updates to ensure accurate predictions in real time.
    \item \textbf{Performance Monitoring:} Continuous monitoring will track the model’s accuracy 
    and detect performance drift, maintaining reliability for fall detection or heart attack prediction.
    \item \textbf{Alerts and Notifications:} Upon detecting a fall or a critical health event, 
    the system will trigger immediate alerts to caregivers, including GPS location if available.
    \item \textbf{Data Management:} Secure data versioning and tracking will support model improvements 
    while ensuring patient privacy and compliance with health data regulations.
    \item \textbf{Optimization and Feedback:} Ongoing model tuning, informed by user feedback 
    and real-world usage, will refine prediction accuracy and optimize the performance 
    for the limited resources of the local processing hardware.
\end{itemize}

This MLOps framework will maintain the efficiency, security, and adaptability of the health monitoring models, 
ensuring timely detection of emergencies and enhancing the overall safety of patients monitored at home.
