The concept of this project is to design and implement a Smart Health Monitoring System 
based on Cloud of Things technologies. The system integrates IoT-enabled wearable sensors 
to continuously track vital signs such as heartbeat, body temperature, and blood pressure. 
Collected data is securely transmitted to the cloud, where it can be stored, processed, 
and made accessible to healthcare professionals and caregivers. 

In addition to real-time monitoring, the project introduces an artificial intelligence component 
that enhances the system's ability to detect and predict critical health events. 
Two possible applications are considered:
\begin{itemize}
    \item \textbf{Fall Detection:} Using computer vision and machine learning algorithms to detect when a patient falls, 
    and automatically sending an alert with the patient's GPS location.
    \item \textbf{Heart Attack Prediction:} Leveraging monitored vital signs and predictive models 
    to identify early indicators of heart attacks, triggering timely alerts and location sharing for rapid intervention.
\end{itemize}

By combining IoT, cloud technologies, and machine learning, this system aims to improve 
patient safety, provide continuous health insights, and enable rapid emergency response.

\section{Problematic}

Healthcare systems today face several challenges. 
Patients with chronic illnesses or conditions requiring continuous observation 
often lack access to reliable monitoring outside of clinical environments. 
Traditional solutions are expensive, limited in scope, and unable to provide real-time 
feedback to healthcare professionals. 

Moreover, emergencies such as falls or heart attacks are often detected too late, 
delaying treatment and endangering patient lives. 
Medical staff are also burdened with increasing patient loads, 
making personalized monitoring difficult to maintain. 

There is therefore a pressing need for smart systems that can continuously track vital signs, 
apply predictive models to anticipate emergencies, and immediately notify caregivers 
with accurate patient data and location.

\section{Context of the Project}

This project is situated at the intersection of Internet of Things (IoT), 
cloud computing, and artificial intelligence. 
Recent advances in wearable sensors and mobile connectivity make it possible 
to capture vital health data outside of hospitals. 
At the same time, cloud services provide the scalability, storage, and computational power 
necessary to process and analyze this data in real time. 

The integration of machine learning into health monitoring opens the door 
to not only detecting anomalies but also predicting critical health conditions 
before they become life-threatening. 
Within this technological context, our project demonstrates how Cloud of Things 
can enable smarter, more responsive healthcare solutions.

\section{Ambitions}

The ambitions of this project are to:
\begin{itemize}
    \item Improve patient safety and quality of life by enabling continuous, non-intrusive monitoring of vital signs.
    \item Provide healthcare professionals with timely, accurate, and actionable insights through cloud-based dashboards.
    \item Reduce the workload on medical staff by automating data collection and emergency alerts.
    \item Integrate GPS-enabled notifications to ensure rapid assistance in case of falls or cardiac emergencies.
    \item Explore the potential of machine learning in healthcare by implementing either a fall detection system 
    or a heart attack prediction model, with the flexibility to evolve as the project progresses.
\end{itemize}

Through these ambitions, the project seeks to demonstrate how the combination of IoT, cloud, 
and AI technologies can transform traditional healthcare monitoring into a proactive, 
intelligent, and life-saving system.
